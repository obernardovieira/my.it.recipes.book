\subsection{mongo}
mongodb é uma base de dados de código aberto, de alto desempenho e orientada a documentos. Escrito em C++.\newline \\
\textbf{Links Uteis}:
\begin{itemize}
\item \url{https://github.com/mongodb/mongo}
\item \url{https://www.mongodb.com/}
\end{itemize}

\subsubsection{Database Manager}
Robomongo é uma ferramenta nativa, de código aberto, utilizada para a manutenção de uma base de dados mongo.\newline \\
\textbf{Links Uteis}:
\begin{itemize}
\item \url{https://github.com/Studio3T/robomongo}
\item \url{https://robomongo.org/}
\end{itemize}
Pode agora criar-se uma base de dados e um utilizadores para controlar essa base de dados, obrigando a uma autenticação prévia, para uso da mesma.\newline
Por padrão, uma base de dados mongo não tem qualquer utilizador nem senha, sendo possível conectar ao servidor mongo e criar uma base de dados e um utilizador para a mesma. É possivel ter várias bases de dados, com utilizadores diferentes com permissões diferentes.
\paragraph{Na primeira utilização}
Utilizando robomongo, na primeira utilização cria-se uma ligação inserindo apenas o endereço (exemplo "localhost:27017").
\paragraph{Utilizador padrão}
Depois da ligação efetuada, à esquerda é apresentado o nome da ligação e o que existe nessa base de dados mongo que pode ser apresentado de acordo com a autenticação. Neste caso é apresentado tudo pois não existe qualquer restrição. Para isso, utilizando a base de dados \textit{admin} cria-se um utilizador com permissões \textit{root}.\newline \\
Sobre o nome da ligação, clica-se com o lado direito do cursor e após escolher a opção \textit{open shell}, usa-se um código como o apresentado de seguida:
\begin{lstlisting}[style=json]
use admin
db.createUser(
  {
    user: "myUserAdmin",
    pwd: "abc123",
    roles: [ { role: "userAdminAnyDatabase", db: "admin" } ]
  }
)
\end{lstlisting}
Posteriormente podem ser criadas bases de dados, com uma autenticação diferente, como por exemplo
\begin{lstlisting}[style=json]
use mysoft
db.createUser(
  {
    user: "myUserAdmin",
    pwd: "abc123",
    roles: [ { role: "dbOwner", db: "mysoft" } ]
  }
)
\end{lstlisting}
\textit{mysoft} é uma base de dados que não existe mas irá ser criada (porém não apresentada na sessão atual) e logo de seguida adicionado um utilizador com o papel de dono da base de dados.\newline \\
Outras formas de criar utilizadores e papéis, podem ser vistos em:
\begin{itemize}
\item \url{https://docs.mongodb.com/manual/tutorial/create-users/}
\item \url{https://docs.mongodb.com/manual/reference/built-in-roles/#built-in-roles}
\end{itemize}
