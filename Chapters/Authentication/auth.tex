\section{oauth}
\begin{chapquote}{Wikipédia, \textit{https://en.wikipedia.org/wiki/OAuth}}
OAuth is an open standard for access delegation, commonly used as a way for Internet users to grant websites or applications access to their information on other websites but without giving them the passwords.
\end{chapquote}

\subsection{Simple}
% TODO stuff

\subsection{MFA}
\begin{chapquote}{Wikipédia, \textit{https://en.wikipedia.org/wiki/Multi-factor\_authentication}}
Multi-factor authentication (MFA) is a method of computer access control in which a user is granted access only after successfully presenting several separate pieces of evidence to an authentication mechanism – typically at least two of the following categories: knowledge (something they know), possession (something they have), and inherence (something they are).
\end{chapquote}

\subsection{2FA}
\begin{chapquote}{Wikipédia, \textit{https://en.wikipedia.org/wiki/Multi-factor\_authentication}}
Two-factor authentication (also known as 2FA) is a method of confirming a user's claimed identity by utilizing a combination of two different components. Two-factor authentication is a type of multi-factor authentication.
\end{chapquote}
É muito comum encontrar-se atualmente aplicações que permitem usar duplo fator de autenticação. Mas como é funciona este sistema e como é que se pode criar um sistema destes?
% TODO stuff