\documentclass{report}
%code
\usepackage{listings}
%images
\usepackage{graphicx}
%quotes
\usepackage{csquotes}
%colors
\usepackage{color}
%textbox
\usepackage{tcolorbox}
%encoding
%--------------------------------------
\usepackage[utf8]{inputenc}
\usepackage[T1]{fontenc}
%--------------------------------------
 
%Portuguese-specific commands
%--------------------------------------
\usepackage[portuguese]{babel}
%--------------------------------------
 
%Hyphenation rules
%--------------------------------------
\usepackage{hyphenat}
\hyphenation{mate-mática recu-perar}
%--------------------------------------

\definecolor{lightgray}{RGB}{201,201,201}

\begin{document}

\title{Bem-vindo ao meu livro}
\author{Bernardo Vieira}

\maketitle

\begin{abstract}
Obrigado a quem ainda não agradeci.
\end{abstract}

\section{Introdução}
Mais tarde escrevo uma introdução.

\chapter{games}
\section{unity3d}


\chapter{communication}
\section{api}
\section{json}

\chapter{storage}
\section{sql}
\subsection{mysql}
\subsection{postgres}

\section{nosql}
\subsection{monogo}
\subsection{neo4j}

\chapter{programming languages}
\section{java}
\subsection{inject parametres on command line}

\section{c}
\subsection{circular dependency}

\section{c++}
\subsection{circular dependency}

\section{assembly}
\subsection{registry (and everything else)}


\chapter{fromeworks}
\section{nodejs}
\section{rails}

\chapter{vitualization}
\section{virtualbox}
\section{docker}
\subsection{add parameters to a docker container}
\begin{displayquote}
\$ docker stop my-container \newline
\$ docker rename my-container my-old-container \newline
\$ docker run --volumes-from=my-old-container --restart always myimage \newline
\$ docker rm my-old-container
\end{displayquote}

\chapter{cloud}
\section{azure}

\chapter{ci/cd}
\section{jenkins}

\chapter{linux}
\section{commands}
for files in `ls *.tex`; do mv \$files `echo \$files | sed s/oldname/newname/g`; done
\section{bash}
\begin{lstlisting}[language=sh]
#!/bin/bash

echo $1
\end{lstlisting}

\chapter{electronics}
\section{components}
\section{arduino}
\subsection{board}
\subsection{shields}

\chapter{text processors}
\section{LaTeX}

\subsection{Escrever código Latex}

\begin{lstlisting}[language=TeX]
\*begin{lstlisting}[language=TeX]
\documentclass{article}
\begin{document}
    Hello World
\end{document}
\*end{lstlisting}
\end{lstlisting}

Para além de latex suporta:\newline
\begin{tcolorbox}
ABAP (R/2 4.3, R/2 5.0, R/3 3.1, R/3 4.6C, R/3 6.10),
ACSL Ada (83, 95), Algol (60, 68), Ant, Assembler
(x86masm), Awk (gnu, POSIX), bash, Basic (Visual),
C (ANSI, Handel, Objective, Sharp), C++ (ANSI, GNU,
ISO, Visual), Caml (light, Objective), Clean, Cobol
(1974, 1985, ibm), Comal 80, csh, Delphi, Eiffel,
Elan, erlang, Euphoria, Fortran (77, 90, 95), GCL,
Gnuplot, Haskell, HTML, IDL (empty, CORBA), inform,
Java (empty, AspectJ), JVMIS, ksh, Lisp (empty, Auto),
Logo, make (empty, gnu), Mathematica (1.0, 3.0),
Matlab, Mercury, MetaPost, Miranda, Mizar, ML,
Modula-2, MuPAD, NASTRAN, Oberon-2, OCL (decorative,
OMG), Octave, Oz, Pascal (Borland6, Standard, XSC),
Perl,PHP, PL/I,Plasm, POV,Prolog, Promela,Python,
R,Reduce, Rexx,RSL, Ruby, S (empty, PLUS), SAS,
Scilab, sh, SHELXL, Simula (67, CII, DEC, IBM),
SQL, tcl (empty, tk), TeX (AlLaTeX, common, LaTeX,
plain, primitive), VBScript, Verilog, VHDL (empty,
AMS), VRML (97), XML, XSLT.
\end{tcolorbox}

\section{markdown}

\section{Conclusão}
Foi um prazer.

\end{document}
