\documentclass{report}
\usepackage{listings}
\usepackage{graphicx}
\usepackage{csquotes}
\usepackage{color}
\usepackage{tcolorbox}
%--------------------------------------
\usepackage[utf8]{inputenc}
\usepackage[T1]{fontenc}
%--------------------------------------

\makeatletter
\renewcommand{\@chapapp}{}% Not necessary...
\newenvironment{chapquote}[2][2em]
  {\setlength{\@tempdima}{#1}%
   \def\chapquote@author{#2}%
   \parshape 1 \@tempdima \dimexpr\textwidth-2\@tempdima\relax%
   \itshape}
  {\par\normalfont\hfill--\ \chapquote@author\hspace*{\@tempdima}\par\bigskip}
\makeatother

\definecolor{lightgray}{RGB}{201,201,201}

\begin{document}

\title{Welcome to my book}
\author{Bernardo Vieira}

\maketitle

\begin{abstract}
Thanks to the people I haven't thanked yet.
\end{abstract}

\section{Introduction}
Lately I will introduce myself.

\chapter{games}
\section{unity3d}


\chapter{communication}
\section{api}
\section{json}

\chapter{storage}
\section{sql}
\subsection{mysql}
\subsection{postgres}

\section{nosql}
\subsection{monogo}
\subsection{neo4j}

\chapter{programming languages}
\section{java}
\subsection{inject parametres on command line}

\section{c}
\subsection{circular dependency}

\section{c++}
\subsection{circular dependency}

\section{assembly}
\subsection{registry (and everything else)}


\chapter{fromeworks}
\section{nodejs}
\section{rails}

\chapter{vitualization}
\section{virtualbox}
\section{docker}
\subsection{Copy volumes from older containers}
\begin{displayquote}
\$ docker stop my-container \newline
\$ docker rename my-container my-old-container \newline
\$ docker run --volumes-from=my-old-container --restart always myimage \newline
\$ docker rm my-old-container
\end{displayquote}
\subsection{Volumes to a specific folder}
\begin{displayquote}
\$ docker run -v path/at/host:path/inside/container/folder my-image \newline
\end{displayquote}
\subsection{Look for volumes}
\begin{displayquote}
\$ docker inspect my-image \newline
\end{displayquote}
\lstinputlisting[language=Bash]{files/docker_volume_mount.sh}

\chapter{cloud}
\section{azure}

\chapter{ci/cd}
\section{jenkins}

\chapter{linux}
\section{commands}
for files in `ls *.tex`; do mv \$files `echo \$files | sed s/oldname/newname/g`; done
\section{bash}
\begin{lstlisting}[language=sh]
#!/bin/bash

echo $1
\end{lstlisting}

\chapter{electronics}
\section{components}
\section{arduino}
\subsection{board}
\subsection{shields}

\chapter{text processors}
\section{LaTeX}

\subsection{Write code in Latex}

\begin{lstlisting}[language=TeX]
\*begin{lstlisting}[language=TeX]
\documentclass{article}
\begin{document}
    Hello World
\end{document}
\*end{lstlisting}
\end{lstlisting}

It also supports:\newline
\begin{tcolorbox}
ABAP (R/2 4.3, R/2 5.0, R/3 3.1, R/3 4.6C, R/3 6.10),
ACSL Ada (83, 95), Algol (60, 68), Ant, Assembler
(x86masm), Awk (gnu, POSIX), bash, Basic (Visual),
C (ANSI, Handel, Objective, Sharp), C++ (ANSI, GNU,
ISO, Visual), Caml (light, Objective), Clean, Cobol
(1974, 1985, ibm), Comal 80, csh, Delphi, Eiffel,
Elan, erlang, Euphoria, Fortran (77, 90, 95), GCL,
Gnuplot, Haskell, HTML, IDL (empty, CORBA), inform,
Java (empty, AspectJ), JVMIS, ksh, Lisp (empty, Auto),
Logo, make (empty, gnu), Mathematica (1.0, 3.0),
Matlab, Mercury, MetaPost, Miranda, Mizar, ML,
Modula-2, MuPAD, NASTRAN, Oberon-2, OCL (decorative,
OMG), Octave, Oz, Pascal (Borland6, Standard, XSC),
Perl,PHP, PL/I,Plasm, POV,Prolog, Promela,Python,
R,Reduce, Rexx,RSL, Ruby, S (empty, PLUS), SAS,
Scilab, sh, SHELXL, Simula (67, CII, DEC, IBM),
SQL, tcl (empty, tk), TeX (AlLaTeX, common, LaTeX,
plain, primitive), VBScript, Verilog, VHDL (empty,
AMS), VRML (97), XML, XSLT.
\end{tcolorbox}

\subsection{No new page when new chapter begins}
\begin{lstlisting}[language=TeX]
\usepackage{etoolbox}
\makeatletter
\patchcmd{\chapter}{\if@openright\cleardoublepage\else\clearpage\fi}{}{}{}
\makeatother
\end{lstlisting}

\section{markdown}

\section{keys}
\subsection{ssh}
\subsection{gpg}

\section{oauth}
\begin{chapquote}{Wikipédia, \textit{https://en.wikipedia.org/wiki/OAuth}}
OAuth is an open standard for access delegation, commonly used as a way for Internet users to grant websites or applications access to their information on other websites but without giving them the passwords.
\end{chapquote}
\subsection{Simple}
\subsection{2FA}
\subsection{MFA}

\section{version control}
\begin{chapquote}{Wikipédia, \textit{https://en.wikipedia.org/wiki/Version\_control}}
A component of software configuration management, version control, also known as revision control or source control, is the management of changes to documents, computer programs, large web sites, and other collections of information.
\end{chapquote}
\subsection{git}
\begin{chapquote}{Wikipédia, \textit{https://en.wikipedia.org/wiki/Git}}
Git is a version control system (VCS) for tracking changes in computer files and coordinating work on those files among multiple people. It is primarily used for software development, but it can be used to keep track of changes in any set of files. As a distributed revision control system it is aimed at speed, data integrity, and support for distributed, non-linear workflows.
\end{chapquote}
\subsubsection{Rebase from another master}
Sometimes our project is a fork of another project. It's really common to happen, especially on GitHub where there is a lot of people forking others projects.\newline
But then, there is a problem, if you cant control de remote repository in your account like it's in your computer, how do you update it? It's easy: 
\begin{lstlisting}[language=Bash]
$ git remote add other <url>
$ git fetch other
$ git rebase other/master master
$ git push
\end{lstlisting}

\section{The End}
It's been a pleasure.

\end{document}
