\documentclass{article}
%code
\usepackage{listings}
%images
\usepackage{graphicx}
%quotes
\usepackage{csquotes}
%colors
\usepackage{color}
%textbox
\usepackage{tcolorbox}
%encoding
%--------------------------------------
\usepackage[utf8]{inputenc}
\usepackage[T1]{fontenc}
%--------------------------------------
 
%Portuguese-specific commands
%--------------------------------------
\usepackage[portuguese]{babel}
%--------------------------------------
 
%Hyphenation rules
%--------------------------------------
\usepackage{hyphenat}
\hyphenation{mate-mática recu-perar}
%--------------------------------------

\definecolor{lightgray}{RGB}{201,201,201}

\begin{document}

\title{Bem-vindo ao meu livro}
\author{Bernardo Vieira}

\maketitle

\begin{abstract}
Obrigado a quem ainda não agradeci.
\end{abstract}

\section{Introdução}
Mais tarde escrevo uma introdução.

\section{LaTeX}

\subsection{Escrever código Latex}

\begin{lstlisting}[language=TeX]
\*begin{lstlisting}[language=TeX]
\documentclass{article}
\begin{document}
    Hello World
\end{document}
\*end{lstlisting}
\end{lstlisting}

Para além de latex suporta:\newline
\begin{tcolorbox}
ABAP (R/2 4.3, R/2 5.0, R/3 3.1, R/3 4.6C, R/3 6.10),
ACSL Ada (83, 95), Algol (60, 68), Ant, Assembler
(x86masm), Awk (gnu, POSIX), bash, Basic (Visual),
C (ANSI, Handel, Objective, Sharp), C++ (ANSI, GNU,
ISO, Visual), Caml (light, Objective), Clean, Cobol
(1974, 1985, ibm), Comal 80, csh, Delphi, Eiffel,
Elan, erlang, Euphoria, Fortran (77, 90, 95), GCL,
Gnuplot, Haskell, HTML, IDL (empty, CORBA), inform,
Java (empty, AspectJ), JVMIS, ksh, Lisp (empty, Auto),
Logo, make (empty, gnu), Mathematica (1.0, 3.0),
Matlab, Mercury, MetaPost, Miranda, Mizar, ML,
Modula-2, MuPAD, NASTRAN, Oberon-2, OCL (decorative,
OMG), Octave, Oz, Pascal (Borland6, Standard, XSC),
Perl,PHP, PL/I,Plasm, POV,Prolog, Promela,Python,
R,Reduce, Rexx,RSL, Ruby, S (empty, PLUS), SAS,
Scilab, sh, SHELXL, Simula (67, CII, DEC, IBM),
SQL, tcl (empty, tk), TeX (AlLaTeX, common, LaTeX,
plain, primitive), VBScript, Verilog, VHDL (empty,
AMS), VRML (97), XML, XSLT.
\end{tcolorbox}


\section{Conclusão}
Foi um prazer.

\end{document}
