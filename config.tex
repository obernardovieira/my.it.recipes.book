\usepackage{graphicx}
\usepackage{subfig}
\usepackage{csquotes}
\usepackage[hidelinks]{hyperref}
\usepackage{mdwlist}
\usepackage[titletoc]{appendix}
\usepackage[backend=biber,sorting=none]{biblatex}
%\addbibresource{references.bib}
\usepackage{fancyhdr}
%glossaries
\usepackage[acronym,shortcuts,nopostdot,style=super,nonumberlist,toc]{glossaries}
\usepackage{glossaries-extra}
%textbox
\usepackage{tcolorbox}
%encoding
\usepackage[utf8]{inputenc}
\usepackage[T1]{fontenc}
%Portuguese-specific commands
\usepackage[portuguese]{babel}
%Hyphenation rules
\usepackage{hyphenat}
\hyphenation{mate-mática recu-perar}
\usepackage{everypage}
%
\usepackage{xstring}
\usepackage{xcolor}
\usepackage{listings}


\makeatletter
\renewcommand
   {\appendixtocname}{Ap\^{e}ndices}
 \renewcommand
   {\appendixpagename}{Ap\^{e}ndices}
 \renewcommand
   {\appendixname}{Ap\^{e}ndice} \let\oriAlph\Alph
 \let\orialph\alph
 \renewcommand{\@resets@pp}
   {\par\@ppsavesec
     \stepcounter{@pps}%
     \setcounter{section}{0}%
     \if@chapter@pp
       \setcounter{chapter}{0}%
       \renewcommand\@chapapp{\appendixname}%
       \renewcommand\thechapter{\@Alph\c@chapter}%
     \else
       \setcounter{subsection}{0}%
       \renewcommand\thesection{\@Alph\c@section}%
     \fi
     \if@pphyper
       \if@chapter@pp
         \renewcommand
           {\theHchapter}
           {\theH@pps.\oriAlph{chapter}}%
       \else
         \renewcommand
           {\theHsection}
           {\theH@pps.\oriAlph{section}}%
       \fi
       \def\Hy@chapapp
          {appendix}%
     \fi
   \restoreapp
  }
\makeatother





\makeatletter
\renewcommand{\@chapapp}{}% Not necessary...
\newenvironment{chapquote}[2][2em]
  {\setlength{\@tempdima}{#1}%
   \def\chapquote@author{#2}%
   \parshape 1 \@tempdima \dimexpr\textwidth-2\@tempdima\relax%
   \itshape}
  {\par\normalfont\hfill--\ \chapquote@author\hspace*{\@tempdima}\par\bigskip}
\makeatother



\pagestyle{fancy}
\fancyhf{}
\fancyhead[LE,RO]{\leftmark}
\fancyfoot[LE,RO]{\thepage}
\renewcommand{\headrulewidth}{0.2pt}
\renewcommand{\footrulewidth}{0pt}


%%%% JSON style (begin)
\newcommand\JSONnumbervaluestyle{\color{blue}}
\newcommand\JSONstringvaluestyle{\color{red}}

% switch used as state variable
\newif\ifcolonfoundonthisline

\makeatletter

\lstdefinestyle{json}
{
  showstringspaces    = false,
  keywords            = {false,true},
  alsoletter          = 0123456789.,
  morestring          = [s]{"}{"},
  stringstyle         = \ifcolonfoundonthisline\JSONstringvaluestyle\fi,
  MoreSelectCharTable =%
    \lst@DefSaveDef{`:}\colon@json{\processColon@json},
  basicstyle          = \ttfamily,
  keywordstyle        = \ttfamily\bfseries,
}

% flip the switch if a colon is found in Pmode
\newcommand\processColon@json{%
  \colon@json%
  \ifnum\lst@mode=\lst@Pmode%
    \global\colonfoundonthislinetrue%
  \fi
}

\lst@AddToHook{Output}{%
  \ifcolonfoundonthisline%
    \ifnum\lst@mode=\lst@Pmode%
      \def\lst@thestyle{\JSONnumbervaluestyle}%
    \fi
  \fi
  %override by keyword style if a keyword is detected!
  \lsthk@DetectKeywords%
}

% reset the switch at the end of line
\lst@AddToHook{EOL}%
  {\global\colonfoundonthislinefalse}

\makeatother
%%%% JSON style (end)


\bibliography{Chapter/references}
